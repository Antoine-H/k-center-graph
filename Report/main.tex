\documentclass[sigconf]{acmart}

\renewcommand{\baselinestretch}{1.02}

\usepackage{hyperref}
%\usepackage[margin=1in]{geometry}
\usepackage{latexsym,graphicx,amssymb}
\usepackage{amsfonts, amssymb,amsthm}
%\usepackage{amsmath,enumerate}
%\usepackage{wrapfig}
%\usepackage{subfigure}
%\usepackage{float}
%\usepackage{psfig}
%\usepackage{epsfig}
\usepackage{xspace}
\usepackage{paralist}
%\usepackage{times}
%\usepackage[compact]{titlesec}
%\usepackage[boxed,lined]{algorithm2e}
%\usepackage{enumerate}
%\usepackage{cases}
%\usepackage{caption}
\usepackage{url}
%\usepackage{tikz}
%\usetikzlibrary{arrows}
%\usetikzlibrary{topaths,calc}
%\usepackage{booktabs}
\usepackage{mathtools}
\usepackage[autolanguage,np]{numprint}

%\usepackage{flushend}  This causes an error sometimes.

\usepackage{color}
%\newcommand{\maxi}[1]{\textcolor{magenta}{#1}}
\newcommand{\maxi}[1]{\textcolor{magenta}{#1}}
\newcommand{\tocheck}[1]{\textcolor{cyan}{#1}}
%\definecolor{Darkblue}{rgb}{0,0,0.4}
%\definecolor{Brown}{cmyk}{0,0.81,1.,0.60}
%\definecolor{Purple}{cmyk}{0.45,0.86,0,0}
%%\usepackage[breaklinks]{hyperref}
%\hypersetup{colorlinks=true,%pdfborder={1 1 1 [3]},%
%citebordercolor={.6 .6 .6},linkbordercolor={.6 .6 .6},%
%citecolor=black,urlcolor=black,linkcolor=black,pagecolor=black}
%\newcommand{\lref}[2][]{\hyperref[#2]{#1~\ref*{#2}}}

\usepackage{algorithm}
\usepackage[noend]{algpseudocode}
\algloopdefx[loop]{Iterate}
[1][]{{\bf iterate} }

\usepackage{comment}

%\newenvironment{proof}{{\bf Proof:  }}{\hfill\rule{2mm}{2mm}}
%\newenvironment{proofof}[1]{{\bf Proof of #1:  }}{\hfill\rule{2mm}{2mm}}

%\numberwithin{figure}{section}
%\numberwithin{equation}{section}
%\newtheorem{theorem}{Theorem}[section]
%
%%\newtheorem{remark}[theorem]{Remark}
%\newtheorem{corollary}[theorem]{Corollary}
%\newtheorem{lemma}[theorem]{Lemma}
%
%\newtheorem{claim}[theorem]{Claim}
%\newtheorem{fact}[theorem]{Fact}
%\newtheorem{observation}[theorem]{Observation}

\theoremstyle{definition}
%\newtheorem{definition}[theorem]{Definition}

%\newtheorem{example}[theorem]{Example}

%\newtheorem{proposition}{\hskip\parindent Proposition}[section]
%\newtheorem*{prop}{Proposition}
%\newtheorem*{lem}{Lemma}

%\renewcommand{\theequation}{\thesection.\arabic{equation}}
%\renewcommand{\thefigure}{\thesection.\arabic{figure}}

%\newcommand{\qed}{\hfill$\Box$}

%\renewcommand{\algorithmicrequire}{\textbf{Input:}}
%\renewcommand{\algorithmicensure}{\textbf{Output:}}

\newcommand{\eps}{\epsilon}

\newcommand{\ve}{\varepsilon}
\newcommand{\vp}{\varphi}
\newcommand{\B}{\mathbf{B}}
\newcommand{\R}{\mathbb{R}}
\newcommand{\N}{\mathbb{N}}
\newcommand{\E}{\ensuremath{\mathbb{E}}}
\newcommand{\Z}{\mathbb{Z}}
\newcommand{\p}{\mathcal{P}}
\newcommand{\sse}{\subseteq}
\newcommand{\ceil}[1]{\ensuremath{\lceil #1 \rceil}}
\newcommand{\abs}[1]{\lvert#1\rvert}
\newcommand{\dimV}{\dim_V}
\newcommand{\poly}{\operatorname{poly}}
\newcommand{\diam}{\operatorname{diam}}
\newcommand{\spn}{\operatorname{span}}
\newcommand{\TIME}{\ensuremath{\mathsf{TIME}}\xspace}
\newcommand{\TSP}{\ensuremath{\mathsf{TSP}}\xspace}
\newcommand{\TSPN}{\ensuremath{\mathsf{TSPN}}\xspace}
\newcommand{\OPT}{\ensuremath{\mathsf{OPT}}\xspace}
\newcommand{\MST}{\ensuremath{\mathsf{MST}}\xspace}
\newcommand{\T}{\ensuremath{\mathsf{T}}\xspace}

\newcommand{\Exp}{\ensuremath{\mathsf{Exp}}\xspace}
\newcommand{\ALG}{\ensuremath{\mathsf{ALG}}\xspace}
\newcommand{\DP}{\ensuremath{\mathsf{DP}}\xspace}

\newcommand{\W}{\ensuremath{\mathsf{W}}\xspace}
\newcommand{\Wm}{\ensuremath{\mathsf{W}^{-1}}\xspace}

\newcommand{\Wh}{\ensuremath{\mathsf{W}^{\frac{1}{2}}}\xspace}

\newcommand{\Wmh}{\ensuremath{\mathsf{W}^{-\frac{1}{2}}}\xspace}

\newcommand{\Lo}{\ensuremath{\mathsf{L}\xspace}}
\newcommand{\Lc}{\ensuremath{\mathcal{L}}\xspace}

\newcommand{\D}{\ensuremath{\mathcal{D}\xspace}}

\newcommand{\Dc}{\ensuremath{\mathcal{D}}\xspace}

\newcommand{\ray}{\ensuremath{\mathsf{R}}\xspace}
\newcommand{\rayc}{\ensuremath{\mathcal{R}}\xspace}

\newcommand{\Ito}{It\={o}\xspace}

\newcommand{\norm}[1]{\ensuremath{\| #1 \|}}

\newcommand{\Diam}{\ensuremath{\mathsf{Diam}}}
\newcommand{\expct}[1]{\ensuremath{\text{{\bf E}$\left[#1\right]$}}}

\newcommand{\sgn}{\operatorname{sgn}}

\newcommand{\txts}{\textstyle}
\newcommand{\LP}{\ensuremath{\mathsf{LP}}}
%\newcommand{\DP}{\ensuremath{\mathsf{DP}}}
\newcommand{\LPn}{\ensuremath{\mathsf{LP}_n}}
\newcommand{\LPm}{\ensuremath{\mathsf{LP}_m}}
\newcommand{\LPi}{\ensuremath{\mathsf{LP}_\infty}}
\newcommand{\LDi}{\ensuremath{\mathsf{LD}_\infty}}
\newcommand{\CP}{\ensuremath{\mathsf{CP}}}
\newcommand{\CD}{\ensuremath{\mathsf{CD}}}

\newcommand{\floor}[1]{\ensuremath{\lfloor #1 \rfloor}}

\newcommand{\ra}{\rightarrow}

\newcommand{\Un}{\ensuremath{\mathcal{U}}}

\newcommand{\ts}{\textstyle}
\newcommand{\ds}{\displaystyle}

\newcounter{note}[section]
\renewcommand{\thenote}{\thesection.\arabic{note}}

\newcounter{myLISTctr}
\newcommand{\initOneLiners}{%
\setlength{\itemsep}{0pt}
	\setlength{\parsep }{0pt}
	\setlength{\topsep }{0pt}
	%      \usecounter{myLISTctr}
}
\newenvironment{OneLiners}[1][\ensuremath{\bullet}]
	{\begin{list}
		{#1}
		{\initOneLiners}}
	{\end{list}}

\newcommand{\ignore}[1]{}
\newcommand{\longv}[1]{}

\newcommand{\algo}[1]{\text{\sf #1}}

\newcommand*\samethanks[1][\value{footnote}]{\footnotemark[#1]}

\newcommand{\toappendix}[1]{\refstepcounter{note}$\ll${\sf Move to
Appendix:} {\sf \textcolor{red}{#1}}$\gg${\tiny\bf App~\thenote}}

\newcommand{\mauro}[1]{\refstepcounter{note}$\ll${\sf Mauro's
Comment~\thenote:} {\sf \textcolor{blue}{#1}}$\gg${\tiny\bf MS~\thenote}}

\newcommand{\hubert}[1]{{\footnotesize\color{red}[Hubert: #1]}}

\newcommand{\arnaud}[1]{\refstepcounter{note}$\ll${\sf Arnaud's
Comment~\thenote:} {\sf \textcolor{purple}{#1}}$\gg${\tiny\bf MS~\thenote}}

\newcommand{\FD}{\textsc{FullyDynamic}}
\newcommand{\SW}{\textsc{SlidingWindow}}
\newcommand{\f}{\textsc{FD}}
\newcommand{\s}{\textsc{SW}}

\fancyhead{}

\begin{document}

\title{Fully Dynamic $k$-Center Clustering on graphs}

\author{Name}
\authornote{Note}
\affiliation{%
	\institution{Institution}
	%\streetaddress{P.O. Box 1212}
	%\city{Dublin}
	%\state{Ohio}
	%\postcode{43017-6221}
}
\email{email}


\begin{abstract}
	Static and dynamic clustering algorithms are a fundamental tool in any
	machine learning library. Most of the efforts in developing dynamic machine
	learning and data mining algorithms have been focusing on the sliding
	window model (where at any given point in time only the most recent data
	items are retained) or more simplistic models. However, in many real-world
	applications one might need to deal with arbitrary deletions and
	insertions. For example, one might need to remove data items that are not
	necessarily the oldest ones, because they have been flagged as containing
	inappropriate content or due to privacy concerns. Clustering trajectory
	data might also require to deal with more general update operations.

	We develop a $(2+\epsilon)$-approximation algorithm for the $k$-center
	clustering problem with ``small'' amortized cost under the fully dynamic
	adversarial model. In such a model, points can be added or removed
	arbitrarily, provided that the adversary does not have access to the random
	choices of our algorithm.  The amortized cost of our algorithm is
	poly-logarithmic when the ratio between the maximum and minimum distance
	between any two points in input is bounded by a polynomial, while $k$ and
	$\epsilon$ are constant. Our theoretical results are complemented with an
	extensive experimental evaluation on dynamic data from Twitter, Flickr, as
	well as trajectory data, demonstrating the effectiveness of our approach.
\end{abstract}

%\copyrightyear{2018}
%\acmYear{2018}
%\setcopyright{iw3c2w3}
%\acmConference[WWW 2018]{The 2018 Web Conference}{April 23--27, 2018}{Lyon, France}
%\acmBooktitle{WWW 2018: The 2018 Web Conference, April 23--27, 2018, Lyon, France}
%\acmPrice{}
%\acmDOI{10.1145/3178876.3186124}
%\acmISBN{978-1-4503-5639-8/18/04}

\maketitle

%\input{intro.tex}
\section{RELATED WORK}
	%General case
	In their article from 2004, Demetrescu and Italiano present an algorithm
	for the all pairs shortest paths problem. It works with a general graph
	with non-negative real-valued edge weights and supports any sequence of
	operations for an amortized running time of $O(n^2 log^3 n)$ per update.

	%$O(n^2 log n)$ for increase only. Worth mentioning?

	An algorithm maintaining a distance matrix will work in $\Omega(n^2)$ as
	for each update, at most $n^2$ paths are changed. The above algorithm is
	thus essentially optimal.

	In 2017, Abraham, Chechik and Krinninger came up with an algorithm taking a
	directed weighted graph with no unweighted cycles as input, against an
	adaptive online adversary, working in $O(n^{2 + \frac{2}{3}}
	log^{\frac{4}{3}}n)$.

	%Planar case
	Other results for specific classes of graphs exist. In 1995, Henzinger,
	Klein, Rao and Subramanian found a data structure handling planar graphs in
	$O(n^{\frac{9}{7}} log D)$ per operation plus a preprocessing time of
	$O(n^{\frac{10}{7}})$. An operation being a query, an edge deletion an adge
	addition or chanching lenths. $D$ being an upper bound on the sum of the
	absolute-values of the negative edge-lengths.
	%Let G be an n-node planar directed graph G such that the sum of the
	%absolute-values of the negative edge-lengths is at most D. The
	%preprocessing to create the data structure from G is
	%$O(n^{\frac{10}{7}})$. The space required by the data structure is $O(n)$.
	%The time per operation is $O(n^{\frac{9}{7}} log D)$. The time bound for
	%queries, edge-deletion, and changing lengths is worst-case while the time
	%for adding edges is amortized.


%\input{preliminaries.tex}
%\input{algorithm.tex}
%\input{analysis.tex}
%\input{experiments.tex}
%\input{conclusion.tex}

\bibliographystyle{abbrv}
\bibliography{biblio}
\end{document}

\end{document}

